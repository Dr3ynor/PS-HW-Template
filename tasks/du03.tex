\section*{Úkol 3}
\label{sec:task-3}

Na hladině významnosti 5 \% rozhodněte, zda střední hodnoty (popř. mediány) nárůstů FPS po aplikaci 1.5 patche statisticky významně závisí na typu grafické karty. Posouzení proveďte nejprve na základě explorační analýzy a následně pomocí vhodného statistického testu, včetně ověření potřebných předpokladů. V případě, že se nárůst FPS pro různé grafické karty statisticky významně liší, určete pořadí karet dle středního nárůstu FPS (popř. mediánu nárůstu FPS). 
\textbf{Poznámka:} \textit{Srovnání proveďte pro všechny čtyři typy grafických karet.}

\begin{enumerate}[label=\alph*)]
    \item Daný problém vhodným způsobem graficky prezentujte (vícenásobný krabicový graf, histogramy, q-q grafy). Srovnání okomentujte (včetně informace o případné manipulaci s datovým souborem).
    
    \newpage
    \item Ověřte normalitu a symetrii nárůstů FPS u všech čtyř grafických karet (empiricky i exaktně).
    
    \newpage
    \item Ověřte homoskedasticitu (shodu rozptylů) nárůstů FPS mezi jednotlivými kartami (empiricky i exaktně)
    
    \newpage
    \item Určete bodové a 95\% intervalové odhady střední hodnoty (popř. mediánu) nárůstů FPS pro všechny srovnávané karty. Volbu charakteristik proveďte tak, aby byly v souladu se statistickým testem, který plánujete provést v bodě e). (Nezapomeňte na ověření předpokladů pro použití příslušných intervalových odhadů.)
    
    \newpage
    \item Čistým testem významnosti ověřte, zda je pozorovaný rozdíl středních hodnot (popř. mediánů) nárůstů FPS statisticky významný na hladině významnosti 5 \%. Pokud ano, zjistěte, zda lze některé skupiny karet označit (z hlediska nárůstů FPS) za homogenní, tj. určete pořadí karet dle středních hodnot (popř. mediánů) nárůstů FPS. (Nezapomeňte na ověření předpokladů pro použití zvoleného testu.)
\end{enumerate}

\endinput