\subsection*{Jak identifikovat, zda jsou v datech odlehlá pozorování?}
\label{subsec:how-to-identify-outliers}

\noindent
\textul{Empirické posouzení:}
\begin{itemize}
    \item použití vnitřních (vnějších) hradeb
    \item vizuální posouzení krabicového grafu.
\end{itemize}

\noindent 
Jak naložit s odlehlými hodnotami by měl definovat hlavně zadavatel analýzy (expert na danou problematiku).

\subsection*{Jak ověřit normalitu dat?}
\label{subsec:how-to-verify-normality}

\noindent
\textul{Empirické posouzení:}
\begin{itemize}
    \item vizuální posouzení histogramu,
    \item vizuální posouzení grafu odhadu hustoty pravděpodobnosti,
    \item Q-Q graf,
    \item posouzení výběrové šikmosti a výběrové špičatosti.
\end{itemize}

\noindent
\textul{Exaktní posouzení:}
\begin{itemize}
    \item testy normality (např. Shapirův – Wilkův test, Andersonův-Darlingův test, Lillieforsův test, ...)
\end{itemize}

\subsection*{Jak ověřit homoskedasticitu (shodu rozptylů)?}
\label{subsec:how-to-verify-homoscedasticity}

\noindent
\textul{Empirické posouzení:}
\begin{itemize}
    \item poměr největšího a nejmenšího rozptylu,
    \item vizuální posouzení krabicového grafu.
\end{itemize}

\noindent
\textul{Exaktní posouzení:}
\begin{itemize}
    \item F – test (parametrický dvouvýběrový test),
    \item Bartlettův test (parametrický vícevýběrový test),
    \item Leveneův test (neparametrický test).
\end{itemize}

\endinput